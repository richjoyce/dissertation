\chapter{Conclusion}
\label{chap:conclusion}

\section{Rapidly Reconfigurable Research Cockpit}

The creation of the Rapidly Reconfigurable Research Cockpit prototype represented a large amount of integration work.
The use of the hand tracker in the virtual environment was a novel use at the beginning of the project, but now LeapMotion (the company behind the hand tracker) has changed focus to develop for the specific use case of hand tracking in virtual reality.
The capacitive touch sensors provided a useful countermeasure to the initially inaccurate registration between the virtual world and real world.
As the technology matured and with the development of the calibration algorithm, the original goal of non-functioning geometric mockups could be realized.
The development of the calibration mechanism proved to be an important step in the development process, as an accurate registration between the physical components and the visual virtual world provided a much more convincing simulation to new and experienced users alike.

\section{Experimental Findings}

The first experiment (Chapter \ref{chap:pointing}: \nameref{chap:pointing}) provided an initial 

The second experiment (Chapter \ref{chap:ph_exp}: \nameref{chap:ph_exp}) performed a Fitts' Law analysis of the technology, and characterized the differences between using the passive haptics and not using any haptic feedback.
It was found that the Fitts' throughput was significantly higher when using the passive haptics (4.25 bps) compared to no haptics (3.76 bps).
Subjects spent significantly more time in the corrective phase of their trajectory without passive haptics, indicating that the passive haptics helps with the final phases of targeting movements.
Self-reported arm fatigue was also found to be lower for subjects who completed the first condition as passive haptics, though scores converged for the second conditions.
The presence survey found a marginally significant increase in presence using passive haptics.
Subjects did strongly indicate their preference for using the passive haptics.

\section{Discusssion}

\section{Future Work}

Although this initial application of the R3C system is to aviation/space vehicle cockpits, any sufficiently complex human-system interface could be designed with this system, such as telerobotics, air traffic control, robotic surgery, etc.
The technology that supports our proof of concept system is rapidly improving, and further iterations of our technology integration will provide higher fidelity and an easier user experience.
