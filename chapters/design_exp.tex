\chapter{Design Evaluation Experiment}

\section{Introduction}

After investigating the technical approach and the benefit to including the passive haptics layer, we seek to investigate the use of the Rapidly Reconfigurable Research Cockpit in a more realistic design evaluation study.
The advantages of using the R3C system would not be useful if it masked defects in a design study.

\section{Experimental Design}

\subsection{Task Design}

The task the subjects were to perform had a number of requirements.
\begin{itemize}
    \item Ability to simulate designs for completing task on touchscreen and R3C setup
    \item Tracking task using a standard attitude indicator display controlled with joystick
    \item Second task that requires use of multiple button to button movements on the instrument
    \item Sufficient workload such that subjects have high but not full workload
\end{itemize}

\subsection{Instrument Design}

\section{Methods}

Subjects were divided into the two groups, TS and VR.

\section{Results}

\subsection{Demographics}

Twenty-three subjects were recruited from the UC Davis engineering undergraduate and graduate student population.
Twelve subjects were placed in the VR group, and the remaining eleven in the TS group.
The mean age was $21.0 (\sigma = 3.14)$, with 19 male and 4 female subjects.
The female subjects were balanced between the two groups.
Most subjects had no flight experience (two were student pilots), and all of the VR group subjects indicated that they had less than one hour of experience using virutal reality headsets.

\subsection{Performance Measures}

\subsection{Design Feedback}

\section{Discussion}

\section{Conclusion}
