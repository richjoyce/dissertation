\chapter{Prototype Design}
\label{chap:prototype}

\section{Overview}

In this chapter the progression of the Rapidly Reconfigurable Research Cockpit prototype is described.
Three major versions of the prototype are outlined here and explained how they evolved from each other.
The first prototype was not used in a formal experiment, but two versions based on the second prototype were used for Experiment 1 (Chapter \ref{chap:pointing}) and 2 (Chapter \ref{chap:ph_exp}).
The final prototype was used in the third experiment (Chapter \ref{chap:de_exp}).

\subsection{First Prototype}

\begin{itemize}
    \item First generation Oculus prototype
    \item No head tracking on visuals
    \item Hand tracking was primitive but worked
    \item No full hand skeleton tracking, just fingertips
    \item Button recognition alogrithm developed
    \item challenges were hand tracking registration and visuals
    \item Sound with button recognition
\end{itemize}

\subsection{Second Prototype}

\begin{itemize}
    \item second generation Oculus prototype
    \item Big upgrade due to head tracking
    \item Hand tracking software upgrade allowed us to track the whole hand
    \item Capacitive touch developed --first as big pads on each button
    \item Capactiive touch sensor pads which allowed measurement of accuracy
    \item Led to development of calibration mechanism
    \item Hand tracking upside down vs right side up led to moving the tracker
\end{itemize}

\subsection{Final Prototype}

\begin{itemize}
    \item Return to pure 3d printed instruments
    \item Calibration with touchscreen
    \item Calibration on a single plane problem
\end{itemize}

\section{Lessons Learned and Future Work}

\begin{itemize}
    \item IR interference
    \item Small amounts of training make a big difference
    \item Conversely, not explaining fully how it works can lead to good performance/feedback
    \item hand tracking difficulties can be very frustrating
\end{itemize}
