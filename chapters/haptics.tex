\chapter{Passive Haptics Experiment}
\label{chap:ph_exp}

\section{Introduction}

Passive haptics is a term that has been used to describe a variety of technologies or techniques to provide the sense of touch to a user of a virtual environment.
It is often defined by its distinction from active haptics, which simulate the sense of touch with energy exchange, typically electromechanical.
A common active haptic technology used in immersive virtual environments is a haptic glove, often utilizing small motors at the fingertips.
In contrast, passive haptics often utilize proxy objects placed in the physical world to co-incide with the virtual environment experience.
The proxy objects can be simple or complex.
They can be colocated and accurate with the virtual world or purposefully designed to trick the user.
In our paper, we utilize a simple colocated passive haptic device and measure its effect on the presence and performance of subjects using a 2D panel in a 3D immersive virtual environment.

The advantages to using a simple passive haptic can be easily understood: less cost and complexity compared to most active haptic solutions.
However, the disadvantage comes with its inflexibility.
Due to their nature, passive haptics often have to be purpose built for a single or limited experience.
While past research has aimed to address this, by either actively positioning a proxy object or simplifying the proxy object to fool the user, our application does not suffer from this limitation.
The motivation for our research comes from the application of designing aerospace cockpits, complex human-machine interfaces where the user is stationed at their workspace.
For the purpose of evaluating a cockpit design, the user does not need a dynamic tactile environment.
Furthermore, many cockpit design processes already create a physical mockup which can provide the passive haptics for this evaluation.

We present our findings in testing passive haptics versus no haptics in an immersive virtual reality environment.
Using a head-mounted display and a hand tracker, the subjects performed the same Fitts' Law style task under these two haptic conditions.
The passive haptics was a flat surface placed at an angle on a desk in front of their seating area.
Their performance on the Fitts' task was recorded as well as their responses to a presence survey, a self reported arm fatigue score and a general questionnaire.

%\subsection{Related Work}

\subsection{Haptics}

Passive haptics has been a topic of research since the early immersive virtual environments.
Robotic passive haptics were used to ameloriate the inflexibility of a proxy object by utilizing a robotic arm to position the proxy object in the bvirtual environment wher ethe user was reaching.[cite]
Insko found increased presence using passive haptics for a maze, and also found that subjects trained with the passive haptics performed better after they were removed than the group that never used them.
Another track of work combined active haptics with passive haptics, using a haptic glove with a physical panel to create mixed haptics.
Again, performance was increased with the haptics, but minimal differences were found between using the mixed haptics and the passive haptics alone.
Similar to our motivation and work, Schiefele et al. replaced a cockpit panel with a flat panel in an immersive head-mounted virtual environment, and found that users could activate buttons and switches in less time with the panel present than without.
We build on this previous work by investigating the effects of passive haptics with the lastest virtual enviroment technology, as well as performing a complete Fitts' Law characterization between no haptics and passive haptics.

While much of the research involving passive haptics indicates an increase in the presence of the user, some have questioned whether active haptics provides benefits.
Pontonnier et al. discovered that subjects had decreased presence ratings in a virtual assembly task when using a haptic glove, versus both a real environment and a virtual environment without haptics.

\subsection{Fitts' Law}

Fitts' originally devised a relationship between movement time and the distance and size of targets for a human performing rapid aimed movements.
This has since become known as Fitts' Law, and later work has refined the index of difficulty (${ID}$) as:

\begin{equation}
    {ID}=\log_2\left(\frac{D}{W}+1\right)
    \label{index_of_difficulty}
\end{equation}
where $D$ is the distance to the target from the starting location and $W$ is the width of the target.
This formula for index of difficulty is known as the Shannons' formulation.

Commonly, the index of difficulty is related to movement time (${MT}$) through a linear regression.
However, in this work we are concerned with the measurement known as throughput (${TP}$).
Throughput has been recommended as the dependent measures for comparisons between experimental conditions.
As the name suggests, it can be thought of as the rate of information the human can input with the particular experimental setup or input device.
It is defined as the index of difficulty over the movement time, and has the units of ``bits per second.''

\begin{equation}
    {TP}=\frac{ID}{MT}
\end{equation}

The use of Fitts' Law as a tool for human-computer interface research began with the research of Card et al. and has continued
The ISO 9241-9 standard was published with guidance on using Fitts' Law as an evaluation of pointing devices.
There have been efforts to standardize the use of Fitts' Law so that results can be compared across literature.
For a 2-dimensional task (where all the buttons exist on a single plane), it is reccomended to use the circle layout as shown in Figure \#.
This layout is referred to as the ``Fitts' circle'' within this article.

\rule{0.75\textwidth}{1pt}
\begin{itemize}
  \item effective width
\end{itemize}
Soukereff et al., among others, recommend the use of the end point location data to calculate the effective width, which is defined as:
\begin{equation}
    W_e = 4.133\sigma
\end{equation}
where $\sigma$ is the standard deviation of the end point positions.
This correction accounts for the performance of the subject, especially on lower index of difficulty conditions where they may aim for the inside edge of a target.
Hence, the use of the effective width provides the index of difficulty for the task that the subject performed, not the task presented to them.
The effective width is calculated per subject per distance and width configuration, and subsquently used in the index of difficultly equation.

\begin{equation}
    {ID}_e=\log_2\left(\frac{D}{W_e}+1\right)
\end{equation}

\rule{0.75\textwidth}{1pt}
\begin{itemize}
  \item fitts law in virtual environments
  \item collect references
\end{itemize}

\rule{0.75\textwidth}{1pt}
\begin{itemize}
  \item fitts law with haptics
  \item collect references
\end{itemize}

\subsection{Presence}

\begin{itemize}
  \item define presence from reference
  \item collect references
\end{itemize}

\subsection{Arm Fatigue}

Despite the ...
Since fatigue exists only as a subjective measure, it can be hard to measure it between subjects, and sometimes even within.
The negative impact of arm fatigue on using virtual environments makes it worth investigating.
The arm fatigue scale used within this experiment is a Borg Rating of Perceived Exertion (RPE) scale that ranges from 6-20.

\begin{itemize}
  \item not much previous work measuring arm fatigue in VE
  \item hard to measure arm fatigue
  \item references to physiology research with scales
  \item mention how we present our results in hope for further investigation in literature
  \item (this sets up discussion point about arm fatigue measure maybe not being reliable)
\end{itemize}

\section{Methods}

\subsection{Experimental Setup}

\begin{itemize}
  \item system objectives and requirements
  \item lightweight HMD
  \item markerless tracking
  \item low cost?
\end{itemize}

The equipment used consists of an Oculus Rift DK2 and a LeapMotion handtracker.
The LeapMotion handtracker was mounted above the working area and pointed down.
Pilot studies indicated that hand tracking from the LeapMotion was improved utilizing the head mounted configuration of the software, but 

A custom calibration scheme was developed for the hand tracker as the initial registration between physical and virtual worlds was not accurate.
However, the LeapMotion software was very precise, so after performing the calibration the registration was kept stable.
The calibration performed a least squares claculation to solve for a transformation matrix between known real world locations and the reported lcoation from the LeapMotion.



\rule{0.75\textwidth}{1pt}
\begin{itemize}
  \item describe technology used
  \item Oculus Rift DK2, specs of screens, capabilities, etc
  \item LeapMotion, specs, mounted rigidly pointed down
  \item Rendering engine, rendering Hz
  \item Recording of hand trajectories
\end{itemize}

The buttons registered a button press when the user moved into a zone around the button.
The circular buttons used in the experiment had a cylindrical zone, projected X in outward from the panel, and X in inward (which was to help with any misalignment).
When a finger was detected to enter the zone, the color of the button would change to indicate to the user that they were in the correct position.
The finger would then have to remain in the zone for 160 milliseconds to register the button press.
This delay is used to inhibit false positivies, as well as simulate the time it takes to physically depress a button.
Once the button press was registered, a clicking sound would be played and the button would restore to its original color.
It is worth noting that this is a naive algorithm for the detection of pressing a button using a hand tracker.


\rule{0.75\textwidth}{1pt}
\begin{itemize}
  \item technical limitiations
  \item leapmotion tracking is not as flawless as marker/magnetic tracking
  \item note some of our workarounds:
  \item low IR environment, minimize reflections
  \item mounted facing down set in VR mode
\end{itemize}


\subsection{Experimental Task}

The subjects were seated at a desk for the experimental task.

The task was performed with two haptic conditions for each subject: ``No Haptics (NH)'' and ``Passive Haptics (PH).''

There were no differences to the task itself or the method of button activation.
The only difference between the conditions was the removal of the physical panel.
The hand tracker remained in the same location, preserving the location of the buttons.
Aside from the variations due to the task, the dimensions of the virtual world were no different for either condition.

\begin{itemize}
  \item room setup, dimensions of table/panel
  \item explain physical differences, no VR differences
\end{itemize}

Subjects performed the ISO-1234 circle for three different distances (20cm, 30cm, 40cm) and five different button widths (5mm, 10mm, 15mm, 20mm, and 25mm).
These configurations were chosen to span a wide range of indices of difficulty (3.2---6.4).
For each configuration of distance and width, subjects had to complete the full pattern of 11 buttons three times consecutively.
Additionally, the distance was kept constant until they had seen all five button widths.
The distances were presented in either smallest to largest or vice versa, which was counterbalanced among subjects.
This set of 15 configurations was repeated for each haptics condition and the order was kept the same within subjects.
The sequence that the two conditions were presented to each subject was also counterbalanced.

\subsection{Experimental Design}

\begin{itemize}
  \item within subjects design
  \item counterbalancing: condition order creates two groups
  \item counterbalancing: order of distances, widths
  \item purposefully did not wait for learned state
\end{itemize}

\subsection{Participants}

Twenty (20) subjects were recruited from the UC Davis engineering student population, both undergraduate and graduate students.
The age range was 19---29 ($M$=22.95, $SD$=3.0) with 16 males and 4 females.
The genders were balanced amongst the counterbalanced groups.
All subjects indicated either less than one hour or no prior experience with virtual reality.

\subsection{Dependent Measures}

The main quantitative measure of performance is the Fitts' Law throughput.
Time of the button presses were recorded as well the entire trajectory data for each movement performed by the subjects.

\begin{itemize}
  \item time of each button press
  \item trajectory information for filtering
  \item arm fatigue rating every other circle
  \item presence questionnaire after each condition
  \item condition comparison questionnaire at end of experiment
\end{itemize}

\subsection{Statistical Methods}

For all of the dependent measures except for the final questionnaire, data was recorded for both haptics conditions for every subject.
To reduce within subject variability, these measures were statistically tested as related samples.
That is to say, the difference between the scores of each condition were used.
The reported condition difference is positive when the measure was higher with Passive Haptics.
For ease of comparison to other literature, the means of each condition are still reported.

Since this is a task with high transfer of skills between conditions, we will also test the sequence for a possible interaction effect with the dependent measures.


\section{Results}

\subsection{Throughput}

An independent ttest was performed on the throughput differences for sequence of conditions.
There was a significant difference (t(18)=-2.9, p=0.008) in the throughput condition difference for the group that performed Passive Haptics as their first condition (PH First, M=0.48, SD=0.36) than the No Haptics First group (M=1.03, SD=0.46).
Post-hoc repeated measures ttests show signifigance in each group between condition.
PH First had a signifigant effect (t(9)=4.16, p=0.002) between Passive Haptics (M=4.70, SD=0.51) and No Haptics (M=4.22, SD=0.53).
NH First had a signifigant effect (t(9)=7.04, p<0.001) between Passive Haptics (M=4.84, SD=0.63) and No Haptics (M=3.81, SD=0.36).

\subsection{Presence}
Presence scores showed no signifigant difference (t(18)=0.85, p=0.408) for the sequence between PH First (M=9.0, SD=12.69) and NH First (M=4.4, SD=11.59).
The repeated measures ttest between conditions was marginally significant (t(19)=2.48, p=0.022) showing an increase in score for Passive Haptics (M=77.7, SD=9.56) and No Haptics (M=71.0, SD=9.70).

\subsection{Arm Fatigue}
Self-reported arm fatigue scores were takend from the maximum of the last two in each condition.
There was a significant difference (t(18)=-4.01, p=0.001) in the arm fatigue condition difference for the group that performed Passive Haptics as their first condition (PH First, M=-3.50, SD=2.12) than the No Haptics First group (M=-0.05, SD=1.71).
PH First had a signifigant effect (t(9)=-5.22, p=0.001) between Passive Haptics (M=12.10, SD=3.51) and No Haptics (M=15.60, SD=2.46).
NH First had no effect (t(9)=-0.09, p=0.93) between Passive Haptics (M=15.20, SD=3.26) and No Haptics (M=15.25, SD=2.66).

\subsection{Condition Comparison}
The condition comparison.

\section{Discussion}

\begin{itemize}
  \item time of each button press
  \item trajectory information for filtering
  \item arm fatigue rating every other circle
  \item presence questionnaire after each condition
  \item condition comparison questionnaire at end of experiment
\end{itemize}

\section{Conclusion}

